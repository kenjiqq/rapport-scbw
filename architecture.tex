\section{Architectures}
Classical games like chess or tic-tac-toe are usually ``solved'' by AIs using a
single approach and searching through a single tree of game states, though
usually by optimizing the search and tree in various ways.

In comparison most approaches to AIs playing real-time strategy games usually
have to use domain knowledge do further subdivide the problem of playing the
game, because of the fine-grained simulations involved, and the various levels
of abstraction that is needed to get a successful AI. And especially when
approaching the way humans think about a problem more complex architectures
are needed.

\subsection{Decomposition of problem}
Michael Buro in his 2003 call for research \cite{buro2003rts} identified six
important sub-problems in real-time strategy games that he said would be
interesting for AI research to focus on:

\begin{description}
  \item [Resource management.] To be able to build up an army one needs to gather
    resources, and the balance between gathering resources (by creating workers),
    building an army and evolving through the technology tree is an important
    part of the macro/high-level strategy.
  \item [Decision making with uncertainty.] Because of fog of war, there is a
    high degree of uncertainty involved in the decision making. Therefore the AI
    needs to create hypotheses and act according to them, and should scout to
    confirm these.
  \item [Spatial and temporal reasoning.] Analyzing and predicting spatially as
    well as temporarily. Identifying choke points and predicting outcomes and
    utilities of actions it takes are some obvious applications.
  \item [Collaboration.] In most RTSes it is possible for players to ally, and
    how to share intelligence and coordinate attacks is a challenging problem,
    though maybe not as interesting yet.
  \item [Opponent modelling.] Learning from the opponent is an important skill,
    and exploting the weaknesses of your opponent is an important aspect of
    human-level playing.
  \item [Adverserial real-time planning.] Abstracting away micro-level
    management to allow for more efficient search in the game state-space, and
    translate the found solutions back, is an important problem to solve.
\end{description}

A lot of research as been done into AIs for RTSes since this, however, and the
list might be a bit outdated. For example, one important aspect of most AIs
today is the micro-management of units, trying to maximize the utility of them 
(maximizing output of resource gatherers and damage dealt by offensive units,
for example).

Another important problem that is under-valued by the above list is learning
from existing knowledge, like learning build-orders from replays of games played
by humans (or other bots, though the utility of that might not be substantial).
This can be integrated into several of the items above, for example the decision
with uncertainty by statistically inferring the most probable states by
learning from earlier games.

A more general and simplified breakdown of the problem of playing Starcraft can
be found in Ben Weber's presentation from the AIIDE 2010 StarCraft AI
Competition:\cite{weber2010aiide}

\begin{description}
  \item [Managing economy] is the same as the resource management mentioned
    above, and is about getting a steady income.
  \item [Expanding the tech tree] to get more powerful and varied units.
  \item [Producing units] is perhaps one of the most complex parts. This
    involves both buildings and movable units, defensive and offensive.
  \item [Attack opponent] usually is not a very explicit action, but can still
    be pretty complex, since one needs to evaluate its own state against what
    it knows about the opponent to know when to attack, and where. This point
    also involves micro-management, which has received a lot of attention from 
    authors of AIs that have ranked highly.
\end{description}

Solving all of the aforementioned problems by themselves are what is the focus
of most research today, but another important problem is tying all of these
solutions together again. This is perhaps one of the most basic, but important,
aspects of the architecture. There are several different ways of doing this,
and some of the most common one is simply sharing a large amount of information
between sub-units in the architecture (for example a black-board based
architecture), or simply having a well-defined graph hierarchy where decisions
are propagated.

\subsection{General architectures}

\subsection{Cognitive Architectures}
\paragraph{Global Workspace Theory}
\paragraph{Cognitive Models in game AIs}
\cite{Arrabales2009}
