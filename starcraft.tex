%!TEX root = main.tex

\section{Starcraft}

\begin{figure}[h!tb]
\centering
\includegraphics[scale=0.5]{graphics/scbw.jpg}
\caption{Star Craft Brood War}
\label{fig:scbwIntro}
\end{figure}

StarCraft is on the surface a very simple game, it has only three different playable races, a
handful of different buildings and units, and relatively simple to comprehend
goals. But once you start analyzing the game, the reality is quite different. The brood war expansion pack was released back in 1998, and has been played at a high level since that and all the way up to today. And the meta game has evolved during the entire lifespan of the game and is still changing today with new tactics showing up from tournament to tournament.
\cite{blizzardstarcraft}

When playing at a really high level you are working with really small windows of opportunity, often called timings. And it is these timings that enable a game with what should in theory be simple elements to have such complex and evolving strategies.

TODO: talk about supply and how it is differnet from race to race. 

\subsection{Terran}

\begin{quote}
Strengths: \\
Mobility, great defense, build anywhere, cloaking, versatility, Marines, strong through whole tech-tree, easy to learn, instant cloak detection, ability to repair buildings and most units. \\
Weaknesses: \\
Tendency to "turtle", need lots of space, require active scouting, require micromanagement for special abilities, vulnerable to Dark Swarm, buildings burn up when highly damaged. 


\cite{terranoverview}
\end{quote}

Terran are the human faction of StarCraft, a futuristic version of man today. They are known for high adaptability with a good variety of defensive and mobile armies. They are best known for their mobile biological armies, or their slow moving turtling tech(tanks) armies that slowly creeps across the map and secures section for section. This versatility makes them a great class with a lot of different possible strategies and combinations that can be effective. 

The Terran worker is the Space Construction Vehicle (SCV). This unit can gather minerals, build buildings and unique for the Terran race repair other mechanical units or buildings. When constructing buildings the unit has to work on the building from the initial placement to the building is complete, meaning it will be unable to perform other action in this time, and can be attacked. If the SCV halts construction or is killed, the building will have to be canceled or finished by another SVC. Like mentioned before it can also repair mechanical units like tanks if they have taken damage, but to perform this action they have to be pulled from other tasks like mining minerals so it is a two edged sword. Terran buildings will slowly self destruct if left at low health, so it is important to repair them if they have taken significant damage. 

Terran buildings also have a unique feature in that they can lift of the ground and fly around after being constructed. They can then land in a new location and continue production of units or upgrades. Some buildings can also create add-ons that unlocks new units and upgrades for that building. Terran also have a unique building in the bunker. This a a defensive building where  biological units can seek refuge while still attacking, but from a fortified position that protects them from damage. The bunker has to be taken out before the units inside can be damaged and killed. While being useless on it's own, the building can be a death trap when filled with infantry. The bunker can also be repaired by an SCV, like any other Terran building, so the SCVs have to be a priority for an attacking army before they can destroy the bunker. 

\subsection{Protoss}
The protoss are an technological advanced alien race that rely on psionic abilities and cybernetics in battle. Because they are the most technological advanced race in StarCraft they are known for their raw power. With powerful but expensive units they can crush their opponents on the battlefield with an outnumbered but superior quality army.

The protoss worker is called a probe, and is like the SCV used to gather minerals and build buildings. Protoss buildings are also not constructed, they are warped in from their home planet, so a probe only needs to place a warp beacon where the building should be placed and then it can return to mining minerals while the building warps in by it self. This allows one probe for construct several buildings at basically the same time, and then return right to mining.
