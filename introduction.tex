%!TEX root = main.tex

\chapter{Introduction}

\section{Background and motivation}
\subsection{StarCraft: Brood War and BWAPI}
StarCraft is one of the most popular real time strategy games in the world. It
is developed by Blizzard Entertainment, and in 1998 they released the expansion
pack Brood War which they had developed together with Saffire.

Since its release it has been widely played in professional tournaments, as
well as been used extensively in research on artificial intelligence, partly
thanks to the BWAPI project which is a project to develop and maintain an API
for developing artificial intelligences that play the game. In addition, this
API is the basis for a yearly competition where people can submit artificial
intelligences developed with this API. This has led to a fairly large number of
AIs being developed, of various degrees of complexity and novelty.

Most architectures are fairly straight-forward and rule-based, with some
novel research into using multi-agent systems which seems to be fairly
successful.

\subsection{Cognitive architectures}
One area that haven't been as well explored in relation to RTSes in general and
StarCraft in particular, is cognitive architectures, architectures inspired by
mental processes. There has been some research done into implementing cognitive
models for use in first-person shooter games, but not into real-time strategy
games. It would seem intuitive that real-time strategy games, which have been
considered a relatively hard problem to solve in a human-like fashion, would
benefit from using models based on our understanding of human cognition.


\section{Goals and Research Question}
We want to explore the use of a cognitive architecture for the use in an AI for
StarCraft: Brood War.

Our research goal is as follows:
\begin{quote}
 ``Develop a novel AI Architecture for StarCraft: Brood War.''
\end{quote}


\section{Contributions}
The following individuals has contributed to the work presented in this report.

\begin{enumerate}
 \item Helge Langseth
 \item Anders Kofod-Petersen
 \item Pauline Haddow
 \item Adam Heinermann
\end{enumerate}


\section{Research Method}
We herped the derp.


\section{Structure}
This paper is structured in four chapters; the introductory chapter you're
currently reading, a chapter on the theory and background for the research, a
chapter on the research results themselves and finally a chapter with the
evaluation of the research and a conclusion.