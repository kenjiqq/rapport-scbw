%!TEX root = main.tex

\chapter{Introduction}

\section{Background and motivation}

\subsection{The Problem}
In 2003 Michael Buro published an article where he requested more artificial
intelligence research in the domain of real-time strategy
games.\cite{buro2003real} Before this, a lot of research was focused mainly on
turn based, real-time board games, like chess and checkers. And a lot of
progress has been done in these fields to the point where they are now able to
beat top level human players in a real-time match. \cite{campbell2002deep} But
these games are both deterministic and fully observable, whereas real-time
strategy games usually are only partially deterministic and partially
observable.

In the wake Buro's call to arms more work has been invested in this area, and
several platforms for RTS research has been used. One platform that has been
used a lot is Wargus\cite{wargus}, a Warcraft 2 clone where they created a
Lua-based AI scripting language for efficient bot creation. But this game had
quite a few limitations on individual management of units, so in recent years
StarCraft: Brood War has been getting a lot more attention as a platform for
building game AI. Several competitions are held each year where implemented AI
solutions can compete with each other and measure their performance. But even
though a lot has happened with the field in recent years, Starcraft AIs still
have ways to go before it can measure up to a human player.\cite{eisbotvsfong}.

Simply winning is however not always the goal, most game-playing artificial
intelligences are made to be realistic and engaging to compete with, so in many
games simply playing well is not enough. According to Arrabales et
al \cite{arrabales2009gamechars} it is still more realistic and engaging to be
playing with other humans than with synthetic agents. So to attempt to lessen
this gap, it could be interesting to make synthetic agents play more human-like,
and to do this one would probably want to look into more biologically inspired
methods.

\subsection{StarCraft: Brood War and BWAPI}
StarCraft is one of the most popular real-time strategy games in the world. It
was developed by Blizzard Entertainment, and in 1998 they released the expansion
pack Brood War. The expansion pack included new maps, units and upgrades for
each of the races in the game.
 
Since its release it has been widely played in professional tournaments, as well
as been used extensively in research on artificial intelligence, thanks to the
BWAPI project which is a free software project aimed at developing and
maintaining an API, named BWAPI, for creating artificial intelligence modules
for Brood War. In addition, this API is the basis for several yearly
competitions where people can submit AI bots that will be pitted against other
bots to measure their relative performance. This has led to a large number of
AIs being developed of various degrees of complexity and novelty, both from
researchers and hobby developers. 

This Brood War Application Programming Interface (BWAPI) is an open source C++
framework project for creating AI modules for StarCraft: Brood War. The API
provides the programmer with information about the game state and the individual
units, as well as allows them execute actions similar to what a normal player
would be able to perform. There are a two different ways to ``inject'' a bot
into StarCraft; it can be made as a module that gets loaded into StarCraft or as
a standalone process that communicates with BWAPI through a shared memory area.
The API also supports several degrees of accessibility, meaning you can play a
game as a normal real player would with limited information, or you can start a
game that is fully observable. The later is mainly of academic interest, where
you want to test an AI that requires a fully observable environment.  

\subsection{The project}
In this project we will familiarize ourselves with the game StarCraft: Brood
War, and what challenges that presents when creating a computer program that
will play the game. We will identify the different aspects of a StarCraft game
that are important to solve in order to create a functioning system, and also
look at how other researchers have solved these problems in their systems.
Ultimately we will select and define an architecture for our system that will
have an modular approach in order to support easier collaboration when
implementing the system. 

In order to get a good overview of existing solutions and map the current state
of the research into this field, we will perform a structured literature review.
This will give us a good overview of the theory behind the state of the art
when it comes to our research problem.

So our project is divided into three main parts:
\begin{enumerate}
  \item Identify the most important aspects of a StarCraft match.
  \item Research existing solution and theories, using a structured literature
review.
  \item Design an architecture for a modular StarCraft playing computer program.
\end{enumerate}

\section{Contributions}
The following individuals has contributed to the work presented in this report.

\begin{itemize}
 \item Helge Langseth, supervisor.
 \item Anders Kofod-Petersen, supervisor.
 \item Pauline Haddow, supervisor.
 \item Adam Heinermann, developer of BWAPI, help with setup and pointers for
information regarding it.
\end{itemize}

\section{Report Structure}
This paper is structured in four chapters:
\begin{itemize}
\item Chapter 1: \textbf{Introduction} \\
This chapter describes the motivation and goal of the project as well as who contributed and the general structure of the report.
\item Chapter 2: \textbf{Theory} \\
This chapter presents some theory on Starcraft in general, how the most important mechanics works and the difference between the available races.  and background for the research, a
chapter on the research results themselves and finally a chapter with the
evaluation of the search and design, and a conclusion.
\item Chapter 3: \textbf{Results} \\

\item Chapter 4: \textbf{Evaluation} \\

\end{itemize}