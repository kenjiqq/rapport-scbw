%!TEX root = main.tex

\chapter{Introduction}

\section{Background and motivation}

\subsection{The Problem}
In 2003 Michael Buro published a article where he requested more artificial
intelligence research in the domain of real-time strategy
games.\cite{buro2003real} Before this, a lot of research was focused mainly on
turned based real-time board games, like chess and checkers. And a lot of
progress has been done in these fields to the point where they are now able to
beat top level human players in a real-time match. \cite{campbell2002deep} But
these games are both deterministic and fully observable, whereas real-time
strategy games usually are only partially deterministic and partially
observable.

In the wake Buro's call to arms more work has been invested in this area, and
several platforms for RTS research has been used. One platform that has been
used a lot is Wargus\cite{wargus}, a Warcraft 2 clone where they created a Lua
AI scripting language for efficient bot creation. But this game had quite a few
limitations on individual management of units, so in recent years StarCraft:
Brood War has been getting a lot more attention as a platform for building game
AI. Several competitions are held each year where implemented AI solutions can
compete with each other and measure their performance. But even though a lot has
happened with the field in recent years, Starcraft AIs still have a far way to
go before it can measure up to a human player.\cite{eisbotvsfong}.

\subsection{StarCraft: Brood War and BWAPI}
StarCraft is one of the most popular real-time strategy games in the world. It
was developed by Blizzard Entertainment, and in 1998 they released the expansion
pack Brood War. The expansion pack included new maps, units and upgrades for
each of the races in the game.
 
Since its release it has been widely played in professional tournaments, as well
as been used extensively in research on artificial intelligence, thanks to the
BWAPI project which is a free software project aimed at developing and
maintaining an API, named BWAPI, for creating artificial intelligence modules
for Brood War. In addition, this API is the basis for several yearly
competitions where people can submit AI bots that will be pitted against other
bots to measure their relative performance. This has led to a large number of
AIs being developed, of various degrees of complexity and novelty, both from
researchers and hobby developers. 

The Brood War Application Programming Interface (BWAPI) is a open source C++
framework project that is used for creating AI modules for StarCraft: Brood War.
The API provides the programmer with information about the game state and the
individual units, as well as allows them execute actions similar to what a
normal player would be able to perform. There are a few different ways to create
the bot, it can be made as an module that gets loaded into StarCraft or as a
standalone process that communicates with BWAPI through a shared memory area.
The API also supports several degrees of accessibility, meaning you can play a
game as a normal real player would with limited information, or you can start a
game that is fully observable. The later will be mainly for academic reasons
where you want to test an AI that requires a fully observable environment.  

\subsection{The project}
One area that haven't been as well explored in relation to RTSes in general and
StarCraft in particular, is cognitive architectures, or models of cognition.
There has been some research done into implementing cognitive models for use in
first-person shooter games, but not into real-time strategy games. It would seem
intuitive that real-time strategy games, which have been considered a relatively
hard problem to solve in a human-like fashion, would benefit from using models
based on our understanding of human cognition.

\section{Goals and Research Question}
We want to explore the use of a cognitive architecture for the use in an AI for
StarCraft: Brood War.

Our research goal is as follows:
\begin{quote}
 ``Develop a novel AI Architecture for StarCraft: Brood War.''
\end{quote}


\section{Contributions}
The following individuals has contributed to the work presented in this report.

\begin{enumerate}
 \item Helge Langseth
 \item Anders Kofod-Petersen
 \item Pauline Haddow
 \item Adam Heinermann
\end{enumerate}


\section{Research Method}
We read papers and thought a lot.


\section{Structure}
This paper is structured in four chapters; the introductory chapter you're
currently reading, a chapter on the theory and background for the research, a
chapter on the research results themselves and finally a chapter with the
evaluation of the research and a conclusion.