%!TEX root = main.tex

\chapter{Introduction}

\section{Background and motivation}

\subsection{The Problem}
In 2003 Michael Buro published an article where he requested more artificial
intelligence research in the domain of real-time strategy
games.\cite{buro2003real} Before this, a lot of research was focused mainly on
turn based, real-time board games, like chess and checkers. And a lot of
progress has been done in these fields to the point where they are now able to
beat top level human players in a real-time match. \cite{campbell2002deep} But
these games are both deterministic and fully observable, whereas real-time
strategy games usually are only partially deterministic and partially
observable.

In the wake Buro's call to arms more work has been invested in this area, and
several platforms for RTS research has been used. One platform that has been
used a lot is Wargus\cite{wargus}, a Warcraft 2 clone where they created a
Lua-based AI scripting language for efficient bot creation. But this game had
quite a few limitations on individual management of units, so in recent years
StarCraft: Brood War has been getting a lot more attention as a platform for
building game AI. Several competitions are held each year where implemented AI
solutions can compete with each other and measure their performance. But even
though a lot has happened with the field in recent years, Starcraft AIs still
have ways to go before it can measure up to a human player.\cite{eisbotvsfong}.

Simply winning is however not always the goal, most game-playing artificial
intelligences are made to be realistic and engaging to compete with, so in many
games simply playing well is not enough. According to Arrabales et
al \cite{arrabales2009gamechars} it is still more realistic and engaging to be
playing with other humans than with synthetic agents. So to attempt to lessen
this gap, it could be interesting to make synthetic agents play more human-like,
and to do this one would probably want to look into more biologically inspired
methods.

\subsection{StarCraft: Brood War and BWAPI}
StarCraft is one of the most popular real-time strategy games in the world. It
was developed by Blizzard Entertainment, and in 1998 they released the expansion
pack Brood War. The expansion pack included new maps, units and upgrades for
each of the races in the game.
 
Since its release it has been widely played in professional tournaments, as well
as been used extensively in research on artificial intelligence, thanks to the
BWAPI project which is a free software project aimed at developing and
maintaining an API, named BWAPI, for creating artificial intelligence modules
for Brood War. In addition, this API is the basis for several yearly
competitions where people can submit AI bots that will be pitted against other
bots to measure their relative performance. This has led to a large number of
AIs being developed of various degrees of complexity and novelty, both from
researchers and hobby developers. 

\subsection{The project}
In this project we will familiarize ourselves with the game StarCraft: Brood
War, and what challenges that presents when creating a computer program that
will play the game. We will identify the different aspects of a StarCraft game
that are important to solve in order to create a functioning system, and also
look at how other researchers have solved these problems in their systems.
Ultimately we will select and define an architecture for our system that will
have an modular approach in order to support easier collaboration when
implementing the system. 

In order to get a good overview of existing solutions and map the current state
of the research into this field, we will together with the rest of the Starcraft AI group at IDI NTNU\footnote{Department of Computer and Information Science, Faculty of Information Technology, Mathematics and Electrical Engineering at the Norwegian University of Science and Technology (NTNU) in Trondheim, Norway.} perform a structured literature review to gain insight into the theory behind the current state of the art
when it comes to our research problem.

So our project is divided into three main parts:
\begin{enumerate}
  \item Learn the mechanics of Starcraft Brood War so we can know the most important aspects of a match.
  \item Research existing solution and theory, using a structured literature review.
  \item Design an architecture for a modular StarCraft playing computer agent.
\end{enumerate}


\section{Contributions}
For the structured literature review we collaborated with Magnus Sellereite
Fjell, Stian Veum M{\o}llersen, Tobias Laupsa Nilsen, J{\o}rgen B{\o}e Svendsen,
Espen Auran Rathe, Aleksander Lun{\o}e Waage, {\O}ystein Samuelsen, Finn Robin
K{\aa}veland Hansen, Dag-{\O}yvind Tornes and Jan Eriksson.

We are also grateful for the feedback and cooperation with everyone on the
\#BWAPI IRC channel on QuakeNet, and especially Adam Heinermann who is also a
lead developer on the BWAPI project.

We would also like to thank our supervisors; Helge Langseth, Anders
Kofod-Petersen and Pauline Haddow.

\section{Report Structure}
This report is structured into four chapters:
\begin{itemize}
\item Chapter 1: \textbf{Introduction} \\
This chapter describes the motivation and goal of the project as well as who contributed and the general structure of the report.
\item Chapter 2: \textbf{Theory} \\
This chapter is threefold. First it presents the game of Starcraft, how the most important mechanics works and the difference between the
available races. Then we go over the state of the art when it comes to agents
who play StarCraft, both how the problems they solve are partitioned, as well
as their overall architecture. Lastly we present some background on the current state of cognitive research and models
utilizing this.
\item Chapter 3: \textbf{Results} \\
Here we present our results; our novel architecture for an agent for playing
StarCraft: Brood War, based on the cognitive models we explored earlier.
\item Chapter 4: \textbf{Evaluation} \\
Here we summarize and evaluate the work presented in this report.


\end{itemize}