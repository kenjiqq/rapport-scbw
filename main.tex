% Set a sane document class, 10pt font, and a report template
\documentclass[a4paper, twoside, openright, 12pt]{report}


% Import used packages
\usepackage{graphicx}
\usepackage{hyperref}
\usepackage{listings}
\usepackage{longtable}
\usepackage{lscape}
\usepackage{parskip}
\usepackage{color}
\usepackage{multirow}
\usepackage[lmargin=25mm,rmargin=25mm,tmargin=40mm,bmargin=30mm]{geometry}
\usepackage{setspace}
\usepackage{fancyhdr}

% Bibliographies
\usepackage[defernumbers=true]{biblatex}
\bibliography{slr-scbw/bib/bib}
\bibliography{bibliography}

% Use UTF-8
%\usepackage[utf8x]{inputenc}
\def \authors {Ken B\o{}rge Melhus Viktil \& Martin Tobias Holmedahl Sandsmark}
\def \papertitle {An architecture for an agent playing StarCraft: Brood War}

% Meta-information for the PDF
\hypersetup{
pdfauthor = \authors,
pdftitle = \papertitle,
pdfsubject = {Pre-project for IDI},
pdfkeywords = {project, cognitive, architecture, starcraft, artificial
    intelligence},
pdfcreator = {LaTeX with hyperref package lol},
pdfproducer = {pdflatex}}

%opening
\title{\papertitle}
\author{\authors}

\pagestyle{fancy}

\begin{document}


\begin{titlepage}
\begin{center}

\vspace*{8cm}
\hrule height 1pt
\vspace{.5cm}
\huge{\papertitle}

\vspace{.5cm}
\large{\authors}
\vspace{.5cm}
\hrule height 1pt

\vspace{6cm}
\end{center}
\normalsize
\begin{table}[!h]
\begin{tabular}{ll}
\multirow{4}{*}{\includegraphics[width=20mm]{graphics/logo.png}} & \\
& Department of Computer and Information Science \\
& Faculty of Information Technology, Mathematics and Electrical Engineering \\
& Norwegian University of Science and Technology \\
\end{tabular}
\end{table}
\vspace{.5cm}
\begin{center}
\today
\end{center}
\end{titlepage}
\pagenumbering{roman}
% Insert an empty page
\newpage
\thispagestyle{empty}
\mbox{}

\begin{abstract}
We present an overview of the most important aspect of the game StarCraft as
pertaining to designing a computer program that can play it using artificial
intelligence methods. We then present existing approaches to agent architectures
for playing StarCraft, and an analysis of these. We also present an overview of
an established architecture for simulating cognitive models, which has earlier
been used in robotics and first-person shooter games. Finally we present our own
architecture for an agent playing StarCraft, inspired by the aforementioned
cognitive architecture.

This we hope can be used for the final project with the goal of developing an
agent that can represent the Norwegian University of Technology and Science in
an international competition against other agents.
\end{abstract}


\pagenumbering{roman}
\tableofcontents

\listoffigures

%\listoftables

\pagenumbering{arabic}
%!TEX root = main.tex

\chapter{Introduction}
In this chapter we introduce our project, as well as the background and motivation for doing this. Section \ref{sec:background} presents the background and our motivation for the project, a short introduction to Starcraft and the API we will use and describes what we will do in this project and our main goals. Section \ref{sec:contributions} contains the main contributors to the project as well as our supervisors. \ref{sec:structure} introduces the structure of this report.
\section{Background and motivation}
\label{sec:background}
\subsection{The Problem}
In 2003 Michael Buro published an article where he requested more artificial
intelligence research in the domain of real-time strategy
games.\cite{buro2003real} Before this, a lot of research was focused mainly on
turn based, real-time board games, like chess and checkers. A lot of
progress has been done in these fields to the point where they are now able to
beat top level human players in a real-time match. \cite{campbell2002deep} But
these games are both deterministic and fully observable, whereas real-time
strategy games usually are only partially deterministic and partially
observable, which makes for much more interesting problems.

In the wake Buro's call to arms more work has been invested in this area, and
several platforms for RTS research has been used. One platform that has been
used a lot is Wargus\cite{wargus}, a clone of Blizzard's Warcraft 2, where they
created a Lua-based AI scripting language for efficient artificial game-playing
agent creation. But this game had quite severe limitations on individual
management of units, so in recent years StarCraft: Brood War has been getting a
lot more attention as a platform for experimenting with game playing agents.
Several competitions are held each year where implemented AI agents can
compete with each other and measure their performance. But even though a lot has
happened with the field in recent years, Starcraft agents still have ways to go
before they can measure up to a human player.\cite{eisbotvsfong}.

Simply winning is however not always the goal, most game-playing artificial
intelligences are made to be realistic and engaging to compete with, so in many
situations simply playing well is not enough. According to Arrabales et
al \cite{arrabales2009gamechars} it is still more realistic and engaging to be
playing with other humans than with synthetic agents. So to attempt to lessen
this gap, it could be interesting to make synthetic agents play more human-like,
and to do this one would probably want to look into more biologically inspired
methods, for example inspired by cognitive architectures.

Cognitive architectures have proven to lead to human-like behaviour and choices
in both games\cite{arrabales2009gamechars} and general problem
solving\cite{franklin2003interacting}, and the logical conclusion therefore
seems to be to try to apply these models to StarCraft.

\subsection{StarCraft: Brood War and BWAPI}
\label{sec:scbw}
StarCraft is one of the most popular real-time strategy games in the world. It
was developed by Blizzard Entertainment, and in 1998 they released the expansion
pack Brood War. The expansion pack included new maps, units and upgrades for
each of the races in the game.
 
Since its release it has been widely played in professional tournaments, as well
as been used extensively in research on artificial intelligence, thanks to the
BWAPI project which is a free software project aimed at developing and
maintaining an API, named BWAPI, for creating artificial intelligence modules
for Brood War. In addition, this API is the basis for several yearly
competitions where people can submit AI bots that will be pitted against other
bots to measure their relative performance. This has led to a large number of
AIs being developed of various degrees of complexity and novelty, both from
researchers and hobby developers. 

\subsection{The project}
\label{sec:project}
In this project we will familiarize ourselves with the game StarCraft: Brood
War, and what challenges that presents when creating a computer program that
will play the game. We will identify the different aspects of a StarCraft game
that are important to solve in order to create a good game-playing agent, and
also look at how other researchers have solved these problems in their agents.
Ultimately we will select and define an architecture for our system that will
have an modular approach in order to support easier collaboration when
implementing the system, based on a cognitive model.

In order to get a good overview of existing solutions and map the current state
of the research into this field, we will together with the rest of the Starcraft AI group at IDI NTNU\footnote{Department of Computer and Information Science, Faculty of Information Technology, Mathematics and Electrical Engineering at the Norwegian University of Science and Technology (NTNU) in Trondheim, Norway.} perform a structured literature review to gain insight into the theory behind the current state of the art
when it comes to our research problem.

So our project is divided into three main parts:
\begin{enumerate}
  \item Learn the mechanics of Starcraft Brood War so we can know the most important aspects of a match.
  \item Research existing solution and theory, using a structured literature review.
  \item Design an architecture for a modular StarCraft playing computer
program, based on a cognitive model.
\end{enumerate}

\section{Contributions}
\label{sec:contributions}
For the structured literature review we collaborated with Magnus Sellereite
Fjell, Stian Veum M{\o}llersen, Tobias Laupsa Nilsen, J{\o}rgen B{\o}e Svendsen,
Espen Auran Rathe, Aleksander Lun{\o}e Waage, {\O}ystein Samuelsen, Finn Robin
K{\aa}veland Hansen, Dag-{\O}yvind Tornes and Jan Eriksson.

We are also grateful for the feedback and cooperation with everyone on the
\#BWAPI IRC channel on QuakeNet, and especially Adam Heinermann who is also a
lead developer on the BWAPI project.

We would also like to thank our supervisors; Helge Langseth, Anders
Kofod-Petersen and Pauline Haddow.

\section{Report Structure}
\label{sec:structure}
This report is structured into four chapters:
\begin{itemize}
\item Chapter 1: \textbf{Introduction} \\
This chapter describes the motivation and goal of the project as well as who contributed and the general structure of the report.
\item Chapter 2: \textbf{Theory} \\
This chapter is threefold. First it presents the game of Starcraft, how the most important mechanics works and the difference between the
available races. Then we go over the state of the art when it comes to agents
who play StarCraft, both how the problems they solve are partitioned, as well
as their overall architecture. Lastly we present some background on the current state of cognitive research and models
utilizing this.
\item Chapter 3: \textbf{Results} \\
Here we present our results; our novel architecture for an agent for playing
StarCraft: Brood War based on the cognitive models we explored in chapter 3.
\item Chapter 4: \textbf{Evaluation} \\
Here we summarize and evaluate the work presented in this report, as well as the research methodology.

\end{itemize}
%!TEX root = main.tex

\chapter{Theory}
In this chapter we will the present theory that is important for our project.
Section \ref{sec:starcrafttheory} presents StarCraft the game, as well as
describes the main mechanics of a StarCraft match and the difference between the
races. Section \ref{sec:stateofai} describes current state of StarCraft AI. It
explains how agents today divides the complex problem of playing StarCraft into
subproblems that are solvable in real-time, as well as gives an overview over
the overall architecture of some of the most successful bots as of Spring
2012. We also explain the BWAPI architecture, and how it utilizes shared memory
and JNI to allow bots written in Java. Section \ref{sec:cogarch} describes the
models of cognition and the theory
behind them and a description of how this architecture can be used.
Section \ref{sec:lida} describes the LIDA model of cognition.
%!TEX root = main.tex

\section{Starcraft}

\begin{figure}[h!tb]
\centering
\includegraphics[scale=0.5]{graphics/scbw.jpg}
\caption{Star Craft Brood War}
\label{fig:scbwIntro}
\end{figure}

StarCraft is on the surface a very simple game, it has only three different playable races, a
handful of different buildings and units, and relatively simple to comprehend
goals. But once you start analyzing the game, the reality is quite different. The brood war expansion pack was released back in 1998, and has been played at a high level since that and all the way up to today. And the meta game has evolved during the entire lifespan of the game and is still changing today with new tactics showing up from tournament to tournament.
\cite{blizzardstarcraft}

When playing at a really high level you are working with really small windows of opportunity, often called timings. And it is these timings that enable a game with what should in theory be simple elements to have such complex and evolving strategies.

TODO: talk about supply and how it is different from race to race. 

\subsection{Terran}

\begin{quote}
Strengths: \\
Mobility, great defense, build anywhere, cloaking, versatility, Marines, strong through whole tech-tree, easy to learn, instant cloak detection, ability to repair buildings and most units. \\
Weaknesses: \\
Tendency to "turtle", need lots of space, require active scouting, require micromanagement for special abilities, vulnerable to Dark Swarm, buildings burn up when highly damaged. 


\cite{terranoverview}
\end{quote}

Terran are the human faction of StarCraft, a futuristic version of man today. They are known for high adaptability with a good variety of defensive and mobile armies. They are best known for their mobile biological armies, or their slow moving turtling tech(tanks) armies that slowly creeps across the map and secures section for section. This versatility makes them a great class with a lot of different possible strategies and combinations that can be effective. 

The Terran worker is the Space Construction Vehicle (SCV). This unit can gather minerals, build buildings and unique for the Terran race repair other mechanical units or buildings. When constructing buildings the unit has to work on the building from the initial placement to the building is complete, meaning it will be unable to perform other action in this time, and can be attacked. If the SCV halts construction or is killed, the building will have to be canceled or finished by another SVC. Like mentioned before it can also repair mechanical units like tanks if they have taken damage, but to perform this action they have to be pulled from other tasks like mining minerals so it is a two edged sword. Terran buildings will slowly self destruct if left at low health, so it is important to repair them if they have taken significant damage. 

Terran buildings also have a unique feature in that they can lift of the ground and fly around after being constructed. They can then land in a new location and continue production of units or upgrades. Some buildings can also create add-ons that unlocks new units and upgrades for that building. Terran also have a unique building in the bunker. This a a defensive building where  biological units can seek refuge while still attacking, but from a fortified position that protects them from damage. The bunker has to be taken out before the units inside can be damaged and killed. While being useless on it's own, the building can be a death trap when filled with infantry. The bunker can also be repaired by an SCV, like any other Terran building, so the SCVs have to be a priority for an attacking army before they can destroy the bunker. 

The health regeneration mechanics for the Terran race are twofold. The medic is used to heal biological units after they have suffered damage in battle, they can also heal other medics, but not them self. And mechanical units can be repaired by pulling SCVs from mining and using them to repair the unit. SCVs can also be effective to use in combat as they can repair the unit while it is taking damage, and several SCVs can repair the same unit at the same time for an increased regeneration rate. 

\subsection{Protoss}
The protoss are an technological advanced alien race that rely on psionic abilities and cybernetics in battle. Because they are the most technological advanced race in StarCraft they are known for their raw power. With powerful but expensive units they can crush their opponents on the battlefield with an outnumbered but superior quality army.

The protoss worker is called a probe, and is like the SCV used to gather minerals and build buildings. Protoss buildings are also not constructed, they are warped in from their home planet, so a probe only needs to place a warp beacon where the building should be placed and then it can return to mining minerals while the building warps in by it self. This allows the use of one probe for construction of several buildings at basically the same time, and then it can return right to mining. This can allow an protoss player to setup remote expansions in a short time with a single probe. 

In contrast to the terran race the protoss can't just build buildings anywhere, they have to place them on a power grid generated by a pylon, the protoss supply building. Pylons also power buildings so if an enemy takes out all the pylons around some production building it can no longer produce any units or research upgrades. The player will then have to build new pylons to power up the building again before he can continue production. 

Special for protoss units and buildings are that they have energy shields that protect them against damage and recharges to full strength over time. In order for an enemy to damage a protoss unit, it has to first deplete the shield that protects it only then can they proceed to damaging the unit itself. For an enemy it is then important to finish of the unit when it's shield is fully depleted or it will regenerate to full strength. 

\subsection{Zerg}
 The zerg are a race of very different creatures that have been united under a central intelligence called the Overmind. In their goal for power, they have been selectively evolved towards being very effective killers. Technologically they are far behind the other races, but they make up for this with a big army size and superior biological properties. 

 The zerg worker unit is called a drone. When construction a building a drone will sacrifice it self and mutates into the building. This means that for every building that the player constructs he looses a worker. In every other way they behave just the same as any other worker. For every race the worker unit is produced from the foundation building, for zerg this is the hatchery. But for zerg every other unit is also created from the hatchery, and any unit buildings that are constructed only unlocks the possibility to create that unit from the hatchery. This is because zerg have a mechanic called larva. They are produced by the hatchery over time, up to a maximum of three at a given time, and they can be morphed into another zerg unit that is unlocked. 

 Similar to protoss the zerg can't just build buildings anywhere, because the buildings are biological they require nourishment to function and has to be build on creep. Creep is produced by the Hatchery and Creep Colonies and then radiates outwards from these onto any fertile ground in the area. The Hatchery is the only building that can be built without creep, as it produces its own nourishment. 

 Zerg units and buildings have a unique ability to regenerate lost health over time if left alone. This means an attacker should finish of an attacked unit or it will come back with full health after a while. This also makes a tactical retreat important for the zerg players, as they can then rest their army to allow them to regenerate back to full health after taking damage in a battle. Burrowing is also a unique ability for the zerg units, and has great synergy with the health regeneration ability. This allows them to burrow in the ground to get out of harms way, and here they can regenerate health without worrying about taking damage. This also allows for ambush tactics by burrowing armies and waiting for the enemy before launching a surprise attack. It can even be used to gain map control by burrowing units in strategic places to scout that area. Only using some form of detector can the enemy see and attack units that have burrowed. 
%!TEX root = main.tex

\section{State of Current Starcraft AI}
\label{sec:stateofai}

Classical games like chess or tic-tac-toe are usually ``solved'' by AIs using a
single approach and searching through a single tree of game states, though
usually by optimizing the search and tree in various ways.

In comparison most approaches to AIs playing real-time strategy games usually
have to use domain knowledge do further subdivide the problem of playing the
game, because of the fine-grained simulations involved, and the various levels
of abstraction that is needed to get a successful AI. And especially when
approaching the way humans think about a problem more complex architectures
are needed.

\subsection{Decomposition of Problem}
Michael Buro in his 2003 call for research \cite{buro2003real} identified six
important sub-problems in real-time strategy games that he said would be
interesting for AI research to focus on. The problem subdivision also dictates
how the architecture has to be structured in order to solve them. It is
therefore important to have a clear view of what the different main sub-problems
are and how they will affect each other.

\begin{description}
  \item [Resource management.] Resources are what is used to create buildings
and units. There are four main things one can spend resources on in starcraft,
that is creating new buildings, researching upgrades, expanding to new bases and
producing units. In order to be an efficient player one has to find a good
balance between all four of these, and perform the right action at the right
time, like allocating more money for units and defenses if one is under attack.
Resource management is a big part of the macro play, and making sure to always
spend the resources efficient will be very important for any StarCraft bot.
  \item [Decision making with uncertainty.] Because fog of war hides all the
parts of the map where the player have no units, there is a high degree of
uncertainty involved in the decision making. Scouting then becomes essential, as
the more information you can gather about the enemies activities the less
uncertainty you have to work with. To deal with uncertainty the AI needs to
create hypotheses about the enemies strategies, and then act according to them.
It should also scout to confirm these hypotheses.
  \item [Spatial and temporal reasoning.] Spatial and temporal reasoning is a
huge part of any game playing agent. Being able to identify important spatial
properties as well as the temporal changes is key for playing at a high level.
Even though this is important it's one of the areas of RTS research that has had
the least progress since Buro's call to arms. Identifying choke points, high and
low ground as well as good routes for troop movement are some of the main
features that a good AI should solve efficiently. 
   \item [Collaboration.] In most RTSes it is possible to play game modes where
you are allied with another player to fight against a team of enemies, in these
modes how to share intelligence and coordinate attacks is a challenging problem,
that has not gotten a lot of focus. Up to this point most research have been
focused on one versus one battles, but this is a key problem that will have to
get more attention in the future. 
  \item [Opponent modeling.] Learning from the opponent is an important skill,
that most of the current state of the art AIs does with very limited success. A
human player needs a few games before he can identify weaknesses and
patterns in the opponents play style. Here todays machine learning methods come
up short, and has nowhere near that capability. Exploiting the weaknesses is an
important aspect of human-level playing.
  \item [Adversarial real-time planning.] Abstracting away micro-level
management to allow for more efficient search in the game state-space, and
translate the found solutions back, is an important problem to solve. Because
agents can not afford to think in micro actions or the search space becomes to
big to handle in real-time. 
\end{description}

A lot of research as been done into AIs for RTSes since this, however, and the
list might be a bit outdated. For example, one important aspect of most AIs
today is the micro-management of units, trying to maximize the utility of them 
(maximizing output of resource gatherers and damage dealt by offensive units,
for example).

Another important problem that is under-valued by the above list is learning
from existing knowledge, like learning build-orders from replays of games played
by humans (or other bots, though the utility of that might not be substantial).
This can be integrated into several of the items above, for example the decision
with uncertainty by statistically inferring the most probable states by
learning from earlier games.

A more general and simplified breakdown of the problem of playing Starcraft can
be found in Ben Weber's presentation from the AIIDE 2010 StarCraft AI
Competition:\cite{weber2010aiide}

\begin{description}
  \item [Managing economy] is the same as the resource management mentioned
    above, and is about getting a steady income.
  \item [Expanding the tech tree] to get more powerful and varied units.
  \item [Producing units] is perhaps one of the most complex parts. This
    involves both buildings and movable units, defensive and offensive.
  \item [Attack opponent] usually is not a very explicit action, but can still
    be pretty complex, since one needs to evaluate its own state against what
    it knows about the opponent to know when to attack, and where. This point
    also involves micro-management, which has received a lot of attention from 
    authors of AIs that have ranked highly.
\end{description}

Solving all of the aforementioned problems by themselves are what is the focus
of most research today, but another important problem is tying all of these
solutions together again. This is perhaps one of the most basic, but important,
aspects of the architecture. There are several different ways of doing this, and
one of the most common ones is simply sharing a large amount of information
between sub-units in the architecture (for example a black-board based
architecture), or having a well-defined graph hierarchy where decisions are
propagated.

\subsection{Architectures of Current Starcraft Agents}
Each year there are several big AI conferences that run Starcraft AI RTS
competitions. The the two biggest are held at the Computational Intelligence in
Games conference (CIG) \footnote{\url{
http://ls11-www.cs.uni-dortmund.de/rts-competition/starcraft-cig2011}} and the
AI and interactive digital entertainment conference (AIIDE)
\footnote{\url{http://www.starcraftaicompetition.com/}}. Here most the the state
of the art bots come together to measure their strength compared to the other
bots. Most of the time playing vs a real player is not something that the bots
can be very effective, so these competitions becomes the only real way for
researchers and developers to test their AIs and get some reasonable feedback on
the level of play they are capable of.

Below we review the overall architectures of some of the most interesting and
highest rated bots from recent competitions. Overmind has in recent years been
the most well-known and one of the most successful bots in these competitions,
however there has been no published papers about how it works, and the source
code to it is not available, we haven't included it here. Since recent
competitions requires source code to be released, however, it hasn't
participated recently either.


\subsubsection{BTHAI}
BTHAI utilizes a multi-agent approach to create a system with high levels of
modularity. Each unit and building is represented as an agent that extends a
more general version of that agent type. So every building is a subclass of a
structurAgent, and every unit is a subclass of an unitAgent. These again extend
a baseAgent, and this creates a hierarchic structure where agents of a similar
type can share logic for behavior and strategy, but supports the option of
extending the agent in order to customize and specialize the behavior of that
specific agent. 

For higher level tasks with as tactics, exploration and building the bot uses
managers. The managers maintains all the agent objects for the different units
that the bot controls. Managers also act as an information provider for the
agents so they can access data and statistics about the current state of the
game, for instance how many units that an attacking force consists of, and how
many are defending the base etc. There is also an exploration manager that
handles everything related to exploration, where enemy units have been
discovered and predictions about where they will move. A build planner decides
what order buildings will be constructed in, and squad commanders handles higher
level tactics for a group of units, like attacking or retreating. These managers
also have an hierarchic structure, so that they can be extended in order to
create a more specialized manager, like a race specific build planer. 

The creation of agent instances are the responsibility of an AgentFactory. This factory makes sure that the correct agent is created for a given unit and that if there exists a specialized agent type for that unit the correct one is created and not a general agent. 

For movement of individual units the bot utilizes a potential field implementation. Agents can decide depending on the situation and the need for precise movements if they want to use the built-in starcraft path-finding or the potential field module. 

Figure \ref{fig:bthaiarch} shows a general overview of the architecture used in BTHAI

% TODO: maybe clear page here if that fits when report is done. 

\begin{figure}[h!tbp]
\centering
\includegraphics[scale=0.8]{graphics/bthai.png}
\caption{BTHAI general architecture}
\label{fig:bthaiarch}
\end{figure}

\subsubsection{BroodwarBotQ}
BroodwarBotQ has taken a more divide-and-conquer approach to the higher levels
of macro control then BTHAI, which means it has separate agent types for the
different tasks that a StarCraft game consists of. Worker manager, bases
manager, production manager, and construction manager are some of the macro
oriented agent types that this bot utilizes. None of the managers controls are
orders any of the others around, so the problem of playing StarCraft is divided
into subtasks that each has a manager that should solve them. But some of the
subtasks are not entirely independent, so in order to resolve conflicts between
the modules BroodwarBotQ uses a arbitrator. This also acts as a mediator between
the macro and micro layers of the architecture.

For this bot unit control is realized using Bayesian units\footnote{Bayesian
unit is the name the authors of BroodwarBotQ has given to agents controlled by
baysian networks.} that strives for completion different goals as ordered by
an goal manager.\cite{synnaeve2011bayesian} One of the main goals for the author
of BroodwarBotQ was improve the intelligence of StarCraft bot, mainly the
ability to predict and change strategy based on what the opponent is planing. So
achieve this they have estimators that based on data extracted from StarCraft
replays using Bayesian models, tries to predict what the opponent are doing and
planing and tries to adapt its own game play according to countering that. 

% TODO: maybe clear page here if that fits when report is done. 

\begin{figure}[h!tbp]
\centering
\includegraphics[scale=0.8]{graphics/bbq.png}
\caption{BroodwarBotQ general architecture}
\label{fig:bbqarch}
\end{figure}

\subsubsection{Skynet}
Skynet uses a more hierarchic approach then most of the other bots, where it has
divided its decision making into three different layers. Where higher level
modules issuing commands to the lower level modules. The strategy layer at the
top manages build order strategies that it gives commands to the tactics layer
underneath to execute. The tactics layer contains all the managers that controls
all the different aspects of the game, like resource management, scouting and
macro management. These managers outputs tasks that have to be completed and
sends them to the task manager in the last layer. In this task layer each of the
tasks in the queue will be given to a specific low level module that knows how
to execute it. This is the only place where commands are sent from the bot to
StarCraft. In addition the bot has a series of situational analysis modules that
that continuously analyses the state of the game, and gives input back to the
decision making layers when it identifies something that needs focus.

% TODO: maybe clear page here if that fits when report is done. 

\begin{figure}[h!tbp]
\centering
\includegraphics[scale=0.8]{graphics/skynet.png}
\caption{Skynet general architecture}
\label{fig:skynetarch}
\end{figure}

\subsubsection{Nova}
Nova\cite{pérezmulti} has an architecture design that is quite close to
BroodwarBotQ, except it uses a blackboard instead of an arbitrator. Nova is
designed as a multi-agent system with managers and agents, and was design with
that idea that having maximum possible information about the enemy at all times
is essential for creating a good AI. Micro management is handled by abstraction,
meaning you have a hierarchic system with general agents on the top, like the
squad agent that is in charge of controlling an entire squad of units, and more
specific agents at the bottom, like the Combat agent that handles fighting for
the individual unit on the battlefield. Where as macro is handled with
divide-and-conquer where a big problem is divided into smaller individual
problems that each have a manager that tries to solve it. 

To achieve the initial goal of maximum information access the bot uses a
blackboard for communication. Here all the modules can post data and intentions,
and the other modules can then read this data. Anyone can read and write here,
and all the information is available to every module that wants it. 

% TODO: maybe clear page here if that fits when report is done. 

\begin{figure}[h!tbp]
\centering
\includegraphics[scale=0.8]{graphics/nova.png}
\caption{Nova general architecture}
\label{fig:novaarch}
\end{figure}


\section{Cognitive Architectures}
Cognitive architectures are architectures that base themselves on some model of
human cognition. There are several competing models of cognition, and one of
the most recent and well-supported is the Global Workspace Theory.

One area that haven't been as well explored in relation to RTSes in general and
StarCraft in particular, is cognitive architectures.
There has been some research done into implementing cognitive models for use in
first-person shooter games, but not into real-time strategy games. There
currently have been no attempts at utilizing cognitive models for playing
Starcraft: BroodWar. It would seem intuitive that real-time strategy games,
which have been considered a relatively hard problem to solve in a human-like
fashion, would benefit from using models based on our understanding of human
cognition. It is peculiar that while so much work has been put into making
believable game-playing agents, very little focus has been on reproducing the
effects of cognitive models.  Arrabales et al argue that this is perhaps
because of poor understanding within the field of classical AI of research into
cognition\cite{arrabales2009gamechars}.

\subsection{Global Workspace Theory}

\begin{figure}[h!tb]
\centering
\includegraphics[scale=1.0]{graphics/globalworkspace.png}
\caption{A schematic of the Global Workspace theory\cite{baars2005gwt}}
\label{fig:gwt}
\end{figure}

Global Workspace Theory is a model of cognition that is very well supported by
experimental data, and is one of the most widely accepted
models.\cite{dehaene2001towards} It has been used to implement processes that
imitate human decision making (for example for solving the problem of assigning
people to jobs in the US Navy).\cite{baars2005gwt}\cite{franklin2003interacting}

It is based around an understanding of the brain as a set of many independently
processing modules, working together by utilizing a shared workspace (hence the
``global workspace''). Every ``cognitive cycle'' all the
processes compete for attention, and a single one gets the
proverbial spotlight shone upon it.\cite{baars2005gwt}

\subsection{Cognitive Models in game AIs}
There have been several more or less successful attempts at implementing models
of cognition into game-playing agents. One of the more recent ones is
CERA-CRANIUM. One of the reasons for using computer games for experiments wrt.
high-level artificial intelligence is that the characteristics of computer
games lend themselves to this, by eliminating noise and uncertainty, and
providing a more or less realistic simulated environment.


\subsection{CERA-CRANIUM}
CERA-CRANIUM intends to implement a general architecture for agents based on
various cognitive architectures, and not tied to any specific model of
cognition. It has already been used to implement a bot that plays Unreal
Tournament 2004 (a first-person shooter game) using a model based on the Global
Workspace Theory, as well as a robot for mapping out an unknown environment.
\cite{arrabales2009ceracranium}

It is based on two major modules:
\begin{description}
 \item [CRANIUM] (Cognitive Robotics Architecture Neurologically Inspired
Underlying Manager) is a tool to create and manage a large amount of
simultanous processes interacting through a shared workspace.
 \item [CERA] (Conscious and Emotional Reasoning Architecture)
 utilizes CRANIUM to create a dynamic control architecture structured in
layers, based on computational models of consciousness.
\end{description}

\subsubsection{CRANIUM}
CRANIUM is basically a software library that can execute thousand of parallel
but coordinated processes.
It is based on the understanding of how the brain works, where specialized
regions process information both from the senses and from other regions, and
the connections between these areas enables the emergence of the global
coordination we see in the brain.\cite{baars2005gwt}

In CRANIUM the various processes/modules are similar to the \textit{demons} in
Dennett's 1992 paper ``Consciousness Explained'', and CRANIUM itself is similar
to a \textit{pandemonium}\cite{dennet1992consciousness}.

The way the various processes collaborate on a shared workspace, by subscribing
to it, then read specific data from it, processing it and finally submitting the
new data back to the workspace is basically a blackboard
system.\cite{nii1986blackboard}

\subsubsection{CERA}
CERA is based around four layers; the sensory-motor services, physical layer,
mission-specific layer and the core layer, based on the services provided by
CRANIUM.

The sensory-motor layer is the most basic, and lowest-level layer, which
provides a uniform interface for sensory input and motoric actuation, physical
or simulated. Each sensor and motor has a service in this layer.

The physical layer wraps the sensory-motor layer, doing some pre-processing of
the sensory data, checking that actuator commands are within safety limits,
etc. It doesn't do semantic information binding, only simple, ``dumb''
pre-processing, though.

The mission-specific layer processes the data from the physical layer,
according to the current missions and sub-goals of those missions, as well as
vital behaviour of the bot (the inherent goals). The strict layering means that
this layer can be modified indendently of the other layers, to account for
various needs depending on the assigned tasks, and accounting for functional
integrity.

The core layer is the highest level, and perform higher cognitive functions,
and it is this layer that is adjusted to implement various cognitive
architectures. It has five core modules, however; attention, status assessment,
preconscious management, memory management and self-coordination. While these
are defined as core modules in CERA, the modular design means that they can be
replaced by a custom set.

The physical and mission-specific layers are inspired by cognitive models of
consciousness, where the various modules compete and collaborate in a shared
workspace. In CERA there is two workspaces; one for searching for the solution
to ``what must be the next content of the agent's conscious perception?'' and
``what must be the next action to execute?''. This differs from traditional AI
control architectures where one only attempts to find the best solution to the
second question, and ignoring the first. \cite{arrabales2009ceracranium} argues
that to successfully answer the second question in a human-like fashion, you
first need to answer the first one.

\subsubsection{Perceptual flow}
Perceptual flow is bottom-up, going from the physical sensors to the CERA core
layer. There are several types of percepts, depending on the level they are
produced in. The \textit{single percepts} are singular quantums of perceptions
directly from the percept pre-processors in the sensor service, \textit{complex
percepts} which are made from several individual single percepts, aggregated by
so-called ``percept aggregators''. These are specialized processors in the
physical layer who are responsible for noticing relations and contexts between
individual single percepts. These two types of percepts are both originating
from processors in the physical layer. From the specialized processors in the
mission-specific layer we get more abstract and complex percepts, who are also
more implementation specific. Some of the most important ones are
\textit{mission percepts}, \textit{complex percepts}, \textit{mismatch
percepts} and \textit{novelty percepts}, which say something about contexts for
the current focus of attention, activation, if some sensors input is not
matching the expected input, and the novelty of certain input.

Single percepts are published to the physical workspace, while combined, complex
percepts are published both to the physical workspace as well as the
mission-specific one, which means it becomes available to the specialized
processors in both layers immediately.

The percepts generated by the processors in the mission-specific layers
get published both to the mission-specific layer, as well as sent to the CERA
core layer. While using a single workspace would be possible, this decoupling
allows for a more modular and general design, where more of the architecture
can be re-used for different problems.

\subsubsection{Action flow}
Action flow is similar to perceptual flow, in that it gets conceptually simpler
the lower you go, and more complex and abstract further up in towards the core
layer. The flow is however top-down. From the processes in the mission-specific
layer comes \textit{mission behaviours}, which are decomposed into
\textit{simple behaviours} by action planners in the physical layer. These are
in turn unpacked into \textit{single actions}, by action pre-processors, which
can be executed by the motor services in the sensory-motor layer.

In the physical layer there is a action dispatcher, which is responsible for
dispatching individual actions to the motor services, according to the order
they come in, and the priority they are assigned from the processors they
originate in; as an example actions from simple behaviours with the highest
priority is executed before other actions from simple behaviours with lower
priority.

The selection of behaviours in the mission-specific layer is in turn controlled
by the core layer, and by mission goals.

\subsubsection{Feedback loops}
Looking at how percepts flow upwards from the sensory-motor layer, and back
down again, we can get a concept of \textit{feedback loops}, where different
kinds of events bounces from different layers in the system.

A feedback loop that only goes to the physical layer is compared to a reflexive
action, while loops that go through to the mission-specific layer and the core
layer respectively are compared to unconsciously performed actions and
higher-level conscious actions.

\subsubsection{Software architecture}
This model has a strong requirement on concurrency and asynchronous input and
output, and the software architecture reflects this. It is based on the
Microsoft Robotics Developer Studio, and the Concurrency and Coordination
Runtime in that, as well as the Decentralized Software Services, to implement a
light-weight distributed service-oriented architecture.

\subsubsection{Knowledge representation}
According to Arrabales\cite{arrabales2009ceracranium}, one of the key problems
in artificial general intelligence is knowledge representation. In CERA-CRANIUM
the knowledge is iteratively processed from the lower levels up to the core
layer, where lower layers contextualize and correlate input for higher level
processes. There is for example a proprioception module \footnote{Proprieception
is the knowledge of relative positions of ones own limbs.} which calculates the
position of exteroceptive sensors (sensors that observe external stimuli).
Knowledge is represented internally by geometrical vectors or integer variables,
referenced by contextual parameters referred to as $J$. Each percept is given a
$J$-index to define the relevant context it pertains to (for example geometric
position, or temporal position).

Handling conflicting and contradictory knowledge (which is very common in not
fully observable scenarious, with for example error-prone sensors) is given two
options; either assigning levels of confidence to the data itself and the
contextual parameters, or generating complex percepts representing the mismatch.

\subsubsection{Workspace modulation}
The core layer and the other layers are linked by the way which the
core layer can adjust the parameters by which the workspaces work. CRANIUM
creates a neural-like environment in which several processes can collaborate
and work towards a common goal, while CERA structures and controls the
collaboration in CRANIUM according to the cognitive architecture it models.

\subsubsection{Activation levels}
Cognitive architectures have a clear distinction betweeen implicit and explicit
processing\cite{atkinson2000consciousness}, and in CERA-CRANIUM all percepts
are by default implicit and get processed unconsciously. A specialized
competition inspired by the pandemonium model mentioned earlier selects the
conscious focus for explicit reasoning. The pandemonium demons are the
percepts, behaviours as well as the specialized processors. These are
constantly assigned an \textit{activation level} by the workspace, based on
input from the core layer.

The activation level is first used to weed out percepts with low activation
levels (to conserve computational power), and then the percepts with the highest
values are iteratively processed until one of them wins, and is selected as
conscious focus.

The specialized processors themselves are assigned more or less computational
power depending on their activation level. For specialized processors the
activation level is based on what input they can process.

At any given time there's also several different behaviours generated, and only
the ones with the highest activation level will be selected and executed.

\subsubsection{Contextualization}
The most important factor for the activation level for behaviours and percepts
is the contextualization criteria, which depends on a $J$-index sent to the
workspace from the core layer. The closer a behaviour or percepts own $J$-value
is to the current context $J$ from the core layer, the higher the activation
parameter it gets assigned. This means that percepts that are close to the
current context, and probably relevant for the problem at hand, will be more
likely to get chosen, and that behaviours that are directed at the current
context focus also will be more likely to be chosen.

Since the contextual bias influences which percepts are created, the perception
process is a very dynamic and active process, and not just a passive data
storage system.

\subsubsection{The Match/Mismatch/Novelty mechanism}
It also uses a mechanism to identify unexpected percepts, expected percepts and
novel percepts, and assigns higher activation values to these. This mechanism
was described by Pentti O. Haikonen in his 2007 book ``Robot Brains: Circuits
and Systems for Conscious Machines''\cite{haikonen2007robotbrains}. As an
example; a percept that doesn't match the predicted data will generate a
mismatch percept with an assigned high activation level because it might
require conscious attention. The core layer can then send a context command
with the appropriate $J$ context index to the workspaces to focus on the
unexpected percept.

\subsubsection{Core layer design}
One of the design goals of CERA-CRANIUM is that the higher cognitive abilities
should be problem domain independent.\cite{arrabales2009ceracranium} This means
that the core-layer is directed by meta-goals, instead of mission-specific ones.
This in turn means that different cognitive architectures (as in the models of
cognition, not software architectures) can be implemented on top of CERA-CRANIUM
in the core layer without having to change the lower levels. An example given by
Arrabales of a meta-goal is discovering abstractions, defined as detecting an
invariant in a variance.

\subsubsection{Goals}
Looking back at the different types of feedback loops (reflexive, unconscious,
and conscious), Arrabales identifies three types of goals;
\textit{basic-goals}, \textit{mission-goals} and \textit{meta-goals}, which are
achieved by their respective loops. While the basic and mission goals can be
kept constant for various cognitive architectures, the meta goals are dependent
on what kind of cognitive architecture the core layer is based on. The core
layer is therefore suited to model in higher cognitive features like emotions,
attention and imagination.

As an example, Arrabales implemented a core layer that continously calculates
a $J$-index and sends it with context commands to the workspaces to direct the
conscious focus. He relates this to the General Workspace Theory, where the
contexts generated are the outlines of the metaphorical ``spotlight'', and to
the Multiple Draft Model, where the core layer is in charge of selecting the
winning version, defined as the reduced set of percepts. His implementation is
rule based, where a set of rules takes in percepts and generates a new
$J$-index. These rules are based on the \textit{meta-goals}, and should
therefore, as explained earlier, be mission and domain independent.
\section{LIDA Framework}
\label{sec:lida}
LIDA is both the name of a cognitive model and a software framework implementing
most parts of the LIDA model.
%!TEX root = main.tex

\chapter{Results}
We believe that using a cognitive architecture will allow for much
more human-like agents, which will lead to engaging and interesting gameplay, we
chose to base our architecture solution on the global workspace theory, and on the
CERA-CRANIUM\cite{arrabales2009ceracranium} model of implementation.

\section{Architecture}
See figure \ref{fig:our-architecture} for a high-level overview of the
different parts of our architecture. It is highly inspired by the CERA-CRANIUM
architecture. We have some different naming in our figure to make the
distinction of the layers clearer, and also make the placement of the various
processes clearer.

\subsection{The Game}
This ``layer'' is the actual game state, which we interact with through BWAPI.
It approaches a physical world, although the simulation is not very detailed or
realistic.

\subsection{Sensors and Controllers}
These are what allows us to observe and control the game.

We suggest simply wrapping the Event objects received as sensor percepts. The
full list of event types are as follows:
\clearpage
\begin{multicols}{2}
\begin{itemize}
    \item MatchStart
    \item MatchEnd
    \item MatchFrame
    \item SendText
    \item ReceiveText
    \item PlayerLeft
    \item NukeDetect
    \item UnitDiscover
    \item UnitEvade
    \item UnitShow
    \item UnitHide
    \item UnitCreate
    \item UnitDestroy
    \item UnitMorph
    \item UnitRenegade
    \item UnitComplete
    \item SaveGame
    \item None 
\end{itemize}
\end{multicols}
From this list we can see that at least the Unit-related events, as well as the
NukeDetect event, are highly relevant percepts. The rest are more relevant to
detect game conditions and not directly relevant to the agent.

\begin{figure}[h!tb]
\centering
\includegraphics[scale=0.8]{graphics/our-architecture.png}
\caption{High-level model of our architecture}
\label{fig:our-architecture}
\end{figure}


In addition we suggest using the state of each observable unit as a percept.
This includes position, health and which team it belongs to, as well as the
state of any special abilities. Polling this as well as other information that
we don't receive information for will have to be done periodically, and then
one can generate a percept when the value changes noticeably.

Controllers we suggest be at the level of individual units and buildings. There
will probably have to be specialized controllers for most kinds of units, as
well as for the different kinds of buildings.

\subsubsection{Abstraction Layer}
The processes here are responsible for turning the raw sensor percepts into
more usable information for the decision making processes higher up, for
example by doing terrain analysis to be able to detect choke points that need
to be defended. A more advanced process here would be something that looks at
scouted buildings, and tries to map this over knowledge about build orders.

Processes here are also responsible for breaking up the decisions and behaviours
made higher up down into more manageable pieces, as well as executing them step
by step (this includes path-finding), and finding optimal positions for
buildings.

An important aspect here is squad movement and handling. This includes
micro-managing units so that they are positioned optimally (units with high
armor and without ranged attacks should be in front, for example, while
vulnerable units with ranged attacks should be further away from ``hot spots''
and enemy units).

\subsubsection{Behaviour Layer}
Processes in this layer are responsible for executing the various kinds of
behaviours on a relatively high level. This includes the basics of selecting
what kind of build orders to go for, how to expand through the tech-tree, what
kinds of units should be grouped together in what squads, and so forth.
Selecting various strategies for micro management might also be implemented here
(maneuvers or special tactics, such has hit-and-runs\footnote{Attacking and
immediately retreating.}, shoot-and-scoots\footnote{Long-range attacks, and
immediately moving before enemy forces arrive.}, overwatch\footnote{One group of
units supporting another group.}, kiting \footnote{Shooting while moving and
keeping distance, usually against enemy forces with inferior range and speed,
allows you to attack enemy units while denying counter-attack}, etc.). The
responsibility of actually executing these moves belongs in the abstraction
layer, however.

Another important aspect here is assessing threats. Because this cognitive
approach restricts us to focus on individual parts of the map, we need to make
sure we aren't oscillating between hot points on the map, while ignoring
several other important ones who are lower rated. One approach to
solving this is to mark threats as ``being dealt with'' when we have selected
and dispatched a response to a threat. This could possibly be integrated into a
single module in the behaviour layer, ``acknowledge-dispatch''.

An important process is opponent modeling, or strategy prediction. It can take
in information about scouted technology and structures, and try to predict what
the opponent is trying to do, using domain specific knowledge and resource
constraints, as well as for example learned knowledge from earlier games.
Mismatches between expected and actual results from this can be represented as
mismatch percepts.

\subsubsection{Core Layer}
We propose using a similar rule-based system as described in CERA-CRANIUM for
representing the meta-goals.

Meta goals that could be implemented are emotions such as curiosity and a
desire to dominate the game, as well as a desire for accumulating resources.
This will guide the behaviour of the lower layers, and the interplay between
these will have to be finely tuned to get a good balance between economy and
military power, defenses and offensive tactics.

If the balance is skewered here it could lead to oscillating behaviour and
resource starvation for processes that aren't included in the hot spots
selected by the most active rules.
%!TEX root = main.tex

\chapter{Evaluation}
In this chapter we conclude our work by looking at the goals defined in the
introduction, and evaluate the results and the structured literature
review. Section \ref{sec:futurework} contains our thoughts on future
implementation and improvements of our solution.


\section{Evaluation of Results}
We believe that using cognitive models is interesting both from a
game-development perspective, for more realistic and engaging gameplay, as well
as from a purely academic viewpoint. To our knowledge noone has yet tried
to apply a cognitive model to the domain of StarCraft, and the structured
literature review did not turn up any papers that indicated this. But it has
shown great promise in the domain of for example first person shooters, so more
research are trial is required before we can tell how well it will really work
in RTS games.

Since we haven't implemented and tested the performance of our proposed
architecture, we have to rely on the testing done by others in different domains
for estimating how efficient our agent will be. It is not given that the
single-focus approach taken by a cognitive architecture will be optimal for
something that would intuitively seem to require thinking about multiple things
at the same time. But we believe that the combination of unconscious processes
and the acknowledge-dispatch approach to handling threats and areas requiring
attention will lead to the agent being able to efficiently handle multi-pronged
problems. It is also the way the human brain functions, so there should not be
any theoretical problems to the single-focus approach.

\subsection{Evaluation of the Structured Literature Review}
A large part of this project has been dedicated to the structured literature
review, henceforth referred to as the \textit{SLR}. While the SLR process went
mostly as planned, with few process-specific problems, the results were hugely
lacking and the process required an extraordinate amount of time.

The key problems we have identified were a lack of experience, lack of
leadership, lack of domain knowledge and a young domain.

The lack of experience should be fairly obvious, as SLR isn't very commonly
used in computer science. We were given an article on SLR that described the
process and need for it, but lack of practical experience among the
participants was very evident. There was for example a disagreement on what
would constitute valid sources for articles, since a lot of relevant literature
was not published in peer-reviewed journals. The lack of
experience with SLR also lead to it taking a long time to concretize the goals
of the review itself into search terms.

The lack of leadership stemmed from being several, independent groups with
independent goals (each group is graded individually). This meant that 12
individuals needed to argue to consensus for all decisions, since noone could
or would take on a leadership role.

Lack of domain knowledge was because a large part of the group had little to no
experience with neither the game itself (StarCraft) nor the design of software
architectures for game-playing agents. This meant that concretizing search
terms took much longer than needed, and that the search terms had to be
decided early on, before people had been able to read up on domain knowledge,
which probably lead to a sub-optimal searching process. This was partially
reflected in the results, because many of the highly rated articles were mostly
useless.

Another problematic area is what we briefly touched on earlier, is that much of
the literature is not published in journals or academically recognized sources.
This is because StarCraft: Brood War agents is a relatively young area of
research, and many of the participants are simply programmers who do this for
fun in their spare time. This means that much of the literature describing the
state of the art simply doesn't exist, or only exists as blog posts or similar.
This makes searching for relevant literature hard when one has to stay within
the relatively strict boundaries defined by the SLR. On the other hand,
research into cognition is also a relatively young interest field, and the
amount of recent and valuable literature regarding it is not impossible for a
single person to sift through, and it would have been more efficient than
splitting up the work like with the SLR. The indication for this is that only a
handful of the relevant papers we based our eventual report on turned up in the
SLR, and all the relevant papers that turned up in the SLR we were familiar
with beforehand.

Appendix \ref{appendix:slrreport} contains the structured literature review.

\section{Conclusion}
We have described the main mechanics of a StarCraft: Brood War match and what
the major problems are for an intelligent agent that wants to play it. We have
also performed a structured literature review to get an overview of relevant
papers for the domain.

We then designed a modular architecture for a computer program that can play the
game StarCraft: Brood War, based on a cognitive architecture, emulating some of
the cognitive processes in the human mind. This was based on an exploration of
the problem of playing StarCraft: Brood War, a study of existing agents, and
also a review of relevant literature and earlier research.

We have also briefly described some of the modules that would need to be
implemented in an agent.

\subsection{Evaluation of Goals}
\subsubsection{Identify the Most Important Aspects of a StarCraft Match}
We have described the game play of a StarCraft: Brood War match, and what the
major problems are for an intelligent agent that wants to play it.
\subsubsection{Research Existing Solutions and Theories, Using a Structured
Literature Review}
The conclusion of the structured literature review lead to a large amount of
high-relevancy papers that needed to be read through, which we did, and then
selected what we thought was relevant to our paper.
\subsubsection{Design an Architecture for a Modular StarCraft Playing Computer
Program}
Our previously described architecture is an attempt to bridge the domain of
real-time strategy games, and StarCraft: Brood War in particular, with the
model of cognitive processing that has been explored earlier.

\section{Future Work}
\label{sec:futurework}
For future work the main most important thing would be to implement this
architecture, and test how viable it is. This will include more closely
detailing the various specialized processors in each layer, as well as
elaborating on the meta-goals for the core layer, and implementing a rule set to
describe these. Tweaking the different processors and how they prioritize will
be very important in order to make the different modules collaborate in a
sufficient manner.  

Learning is also a really important part of playing any strategy game, both
learning strategies in general, this can be done from replays or playing many
games, but also to learn in real-time what the strength and weaknesses of the
opponent are. So Another important extension would be to more tightly integrate
learning into the architecture. Inspiration for how to do this could possibly be
done by looking at how it is implemented in the LIDA
model\cite{franklin2007lida}.
\endrefsection
\printbibliography[segment=0]


%
% Appendix
\chapter{Structured literature review}
Conducted by Magnus Sellereite Fjell, Stian Veum M{\o}llersen, Tobias Laupsa
Nilsen, J{\o}rgen B{\o}e Svendsen, Espen Auran Rathe, Aleksander Lun{\o}e Waage,
Martin Tobias Holmedahl Sandsmark, Ken B{\o}rge Melhus Viktil, {\O}ystein
Samuelsen, Finn Robin K{\aa}veland Hansen, Dag-{\O}yvind Tornes and Jan
Eriksson.

\label{appendix:slrreport}
\section{Introduction}
%!TEX root = main.tex

\chapter{Introduction}
In this chapter we introduce our project, as well as the background and motivation for doing this. Section \ref{sec:background} presents the background and our motivation for the project, a short introduction to Starcraft and the API we will use and describes what we will do in this project and our main goals. Section \ref{sec:contributions} contains the main contributors to the project as well as our supervisors. \ref{sec:structure} introduces the structure of this report.
\section{Background and motivation}
\label{sec:background}
\subsection{The Problem}
In 2003 Michael Buro published an article where he requested more artificial
intelligence research in the domain of real-time strategy
games.\cite{buro2003real} Before this, a lot of research was focused mainly on
turn based, real-time board games, like chess and checkers. A lot of
progress has been done in these fields to the point where they are now able to
beat top level human players in a real-time match. \cite{campbell2002deep} But
these games are both deterministic and fully observable, whereas real-time
strategy games usually are only partially deterministic and partially
observable, which makes for much more interesting problems.

In the wake Buro's call to arms more work has been invested in this area, and
several platforms for RTS research has been used. One platform that has been
used a lot is Wargus\cite{wargus}, a clone of Blizzard's Warcraft 2, where they
created a Lua-based AI scripting language for efficient artificial game-playing
agent creation. But this game had quite severe limitations on individual
management of units, so in recent years StarCraft: Brood War has been getting a
lot more attention as a platform for experimenting with game playing agents.
Several competitions are held each year where implemented AI agents can
compete with each other and measure their performance. But even though a lot has
happened with the field in recent years, Starcraft agents still have ways to go
before they can measure up to a human player.\cite{eisbotvsfong}.

Simply winning is however not always the goal, most game-playing artificial
intelligences are made to be realistic and engaging to compete with, so in many
situations simply playing well is not enough. According to Arrabales et
al \cite{arrabales2009gamechars} it is still more realistic and engaging to be
playing with other humans than with synthetic agents. So to attempt to lessen
this gap, it could be interesting to make synthetic agents play more human-like,
and to do this one would probably want to look into more biologically inspired
methods, for example inspired by cognitive architectures.

Cognitive architectures have proven to lead to human-like behaviour and choices
in both games\cite{arrabales2009gamechars} and general problem
solving\cite{franklin2003interacting}, and the logical conclusion therefore
seems to be to try to apply these models to StarCraft.

\subsection{StarCraft: Brood War and BWAPI}
\label{sec:scbw}
StarCraft is one of the most popular real-time strategy games in the world. It
was developed by Blizzard Entertainment, and in 1998 they released the expansion
pack Brood War. The expansion pack included new maps, units and upgrades for
each of the races in the game.
 
Since its release it has been widely played in professional tournaments, as well
as been used extensively in research on artificial intelligence, thanks to the
BWAPI project which is a free software project aimed at developing and
maintaining an API, named BWAPI, for creating artificial intelligence modules
for Brood War. In addition, this API is the basis for several yearly
competitions where people can submit AI bots that will be pitted against other
bots to measure their relative performance. This has led to a large number of
AIs being developed of various degrees of complexity and novelty, both from
researchers and hobby developers. 

\subsection{The project}
\label{sec:project}
In this project we will familiarize ourselves with the game StarCraft: Brood
War, and what challenges that presents when creating a computer program that
will play the game. We will identify the different aspects of a StarCraft game
that are important to solve in order to create a good game-playing agent, and
also look at how other researchers have solved these problems in their agents.
Ultimately we will select and define an architecture for our system that will
have an modular approach in order to support easier collaboration when
implementing the system, based on a cognitive model.

In order to get a good overview of existing solutions and map the current state
of the research into this field, we will together with the rest of the Starcraft AI group at IDI NTNU\footnote{Department of Computer and Information Science, Faculty of Information Technology, Mathematics and Electrical Engineering at the Norwegian University of Science and Technology (NTNU) in Trondheim, Norway.} perform a structured literature review to gain insight into the theory behind the current state of the art
when it comes to our research problem.

So our project is divided into three main parts:
\begin{enumerate}
  \item Learn the mechanics of Starcraft Brood War so we can know the most important aspects of a match.
  \item Research existing solution and theory, using a structured literature review.
  \item Design an architecture for a modular StarCraft playing computer
program, based on a cognitive model.
\end{enumerate}

\section{Contributions}
\label{sec:contributions}
For the structured literature review we collaborated with Magnus Sellereite
Fjell, Stian Veum M{\o}llersen, Tobias Laupsa Nilsen, J{\o}rgen B{\o}e Svendsen,
Espen Auran Rathe, Aleksander Lun{\o}e Waage, {\O}ystein Samuelsen, Finn Robin
K{\aa}veland Hansen, Dag-{\O}yvind Tornes and Jan Eriksson.

We are also grateful for the feedback and cooperation with everyone on the
\#BWAPI IRC channel on QuakeNet, and especially Adam Heinermann who is also a
lead developer on the BWAPI project.

We would also like to thank our supervisors; Helge Langseth, Anders
Kofod-Petersen and Pauline Haddow.

\section{Report Structure}
\label{sec:structure}
This report is structured into four chapters:
\begin{itemize}
\item Chapter 1: \textbf{Introduction} \\
This chapter describes the motivation and goal of the project as well as who contributed and the general structure of the report.
\item Chapter 2: \textbf{Theory} \\
This chapter is threefold. First it presents the game of Starcraft, how the most important mechanics works and the difference between the
available races. Then we go over the state of the art when it comes to agents
who play StarCraft, both how the problems they solve are partitioned, as well
as their overall architecture. Lastly we present some background on the current state of cognitive research and models
utilizing this.
\item Chapter 3: \textbf{Results} \\
Here we present our results; our novel architecture for an agent for playing
StarCraft: Brood War based on the cognitive models we explored in chapter 3.
\item Chapter 4: \textbf{Evaluation} \\
Here we summarize and evaluate the work presented in this report, as well as the research methodology.

\end{itemize}

\section{Structured Literature Review}
\input{slr-scbw/sections/about_slr}

\section{Protocol and Procedure}
\input{slr-scbw/sections/protocol}

\section{Search Engines and Search Strings}
\input{slr-scbw/sections/search_engines}

\section{Search Results}
\input{slr-scbw/sections/search_results}

\clearpage



\end{document}
