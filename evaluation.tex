%!TEX root = main.tex

\chapter{Evaluation}
In this chapter we conclude our work by looking at the goals defined in the
introduction, and evaluate the results and the structured literature
review. Section \ref{sec:futurework} contains our thoughts on future
implementation and improvements of our solution.

\section{Evaluation of the Structured Literature Review}
A large part of this project has been dedicated to the structured literature
review, henceforth referred to as the \textit{SLR}. While the SLR process went
mostly as planned, with few process-specific problems, the results were hugely
lacking and the process required an extraordinate amount of time.

The key problems we have identified were a lack of experience, lack of
leadership, lack of domain knowledge and a young domain.

The lack of experience should be fairly obvious, as SLR isn't very commonly
used in computer science. We were given an article on SLR that described the
process and need for it, but lack of practical experience among the
participants was very evident. There was for example a disagreement on what
would constitute valid sources for articles, since a lot of relevant literature
was not published in peer-reviewed journals. The lack of
experience with SLR also lead to it taking a long time to concretize the goals
of the review itself into search terms.

The lack of leadership stemmed from being several, independent groups with
independent goals (each group is graded individually). This meant that 12
individuals needed to argue to consensus for all decisions, since noone could
or would take on a leadership role.

Lack of domain knowledge was because a large part of the group had little to no
experience with neither the game itself (StarCraft) nor the design of software
architectures for game-playing agents. This meant that concretizing search
terms took much longer than needed, and that the search terms had to be
decided early on, before people had been able to read up on domain knowledge,
which probably lead to a sub-optimal searching process. This was partially
reflected in the results, because many of the highly rated articles were mostly
useless.

Another problematic area is what we briefly touched on earlier, is that much of
the literature is not published in journals or academically recognized sources.
This is because StarCraft: Brood War agents is a relatively young area of
research, and many of the participants are simply programmers who do this for
fun in their spare time. This means that much of the literature describing the
state of the art simply doesn't exist, or only exists as blog posts or similar.
This makes searching for relevant literature hard when one has to stay within
the relatively strict boundaries defined by the SLR. On the other hand,
research into cognition is also a relatively young interest field, and the
amount of recent and valuable literature regarding it is not impossible for a
single person to sift through, and it would have been more efficient than
splitting up the work like with the SLR. The indication for this is that only a
handful of the relevant papers we based our eventual report on turned up in the
SLR, and all the relevant papers that turned up in the SLR we were familiar
with beforehand.

Appendix \ref{appendix:slrreport} contains the structured literature review.


\section{Evaluation of Results}
We believe that using cognitive models is interesting both from a
game-development perspective, for more realistic and engaging gameplay, as well
as from a purely academic viewpoint. To our knowledge noone has yet tried
to apply a cognitive model to the domain of StarCraft, and the structured
literature review did not turn up any papers that indicated this. But it has
shown great promise in the domain of for example first person shooters, so more
research are trial is required before we can tell how well it will really work
in RTS games.

Since we haven't implemented and tested the performance of our proposed
architecture, we have to rely on the testing done by others in different domains
for estimating how efficient our agent will be. It is not given that the
single-focus approach taken by a cognitive architecture will be optimal for
something that would intuitively seem to require thinking about multiple things
at the same time. But we believe that the combination of unconscious processes
and the acknowledge-dispatch approach to handling threats and areas requiring
attention will lead to the agent being able to efficiently handle multi-pronged
problems. It is also the way the human brain functions, so there should not be
any theoretical problems to the single-focus approach.

\section{Conclusion}
We have described the main mechanics of a StarCraft: Brood War match and what
the major problems are for an intelligent agent that wants to play it. We have
also performed a structured literature review to get an overview of relevant
papers for the domain.

We then designed a modular architecture for a computer program that can play the
game StarCraft: Brood War, based on a cognitive architecture, emulating some of
the cognitive processes in the human mind. This was based on an exploration of
the problem of playing StarCraft: Brood War, a study of existing agents, and
also a review of relevant literature and earlier research.

We have also briefly described some of the modules that would need to be
implemented in an agent.

\section{Future Work}
\label{sec:futurework}
For future work the main most important thing would be to implement this
architecture, and test how viable it is. This will include more closely
detailing the various specialized processors in each layer, as well as
elaborating on the meta-goals for the core layer, and implementing a rule set to
describe these. Tweaking the different processors and how they prioritize will
be very important in order to make the different modules collaborate in a
sufficient manner.  

Learning is also a really important part of playing any strategy game, both
learning strategies in general, this can be done from replays or playing many
games, but also to learn in real-time what the strength and weaknesses of the
opponent are. So Another important extension would be to more tightly integrate
learning into the architecture. Inspiration for how to do this could possibly be
done by looking at how it is implemented in the LIDA
model\cite{franklin2007lida}.
